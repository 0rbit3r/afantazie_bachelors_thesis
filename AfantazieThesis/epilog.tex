\chapter{Conclusion}

In this thesis we have presented a novel approach to social network content presentation using graph view
instead of the traditional infinite feed.
We discussed the motivation behind such approach and potential benefits it might bring.

We have implemented a proof-of-concept web application called Aphantasia which utilizes such approach including 
a custom \gls{GLA} implementation and a rendering engine based on PIXI.js.
During the development we have solved several technical challenges regarding user management,
server-client communication, hosting, cache busting, and of course graph rendering.

The finished application provides:
\begin{itemize}
    \item Graph view able to render several hundred nodes at once
    \item Dynamic loading ensuring capability to explore large graphs in order of tens of thousands nodes (and potentially much more)
    \item User management including registration and login
    \item User interface, including not just graph view, but also pages with user settings, post creation form, chat
    and rudimentary notifications
    \item Live preview of the graph view
\end{itemize}

Aphantasia is now ready to be used by general public and we are hosting two localized instances at:
\begin{itemize}
    \item \url{https://aphantasia.io}
    \item \url{and https://afantazie.cz}
\end{itemize}

The source code to the application is available at \url{https://github.com/0rbit3r/afantazie_bachelors_thesis}.
\xxx{The application is licensed under .... a license I guess. TODO}

While we did not manage to implement all the features we expected (namely zoom-based dynamic loading and filtering),
we are satisfied with the result and believe that the application is a good starting point for further development.

We also believe that we have proven graph view to be a viable alternative to interact with social media content.
Graph view might not be suitable for everyone but it certainly attracts a niche of users who can appreciate the benefits it brings.