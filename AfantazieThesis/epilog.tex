\chapter{Conclusion}

In this thesis, we have presented a novel approach to social network content presentation using a graph view
instead of the traditional infinite feed.
We discussed the motivation and the potential benefits it might bring.

We have implemented a proof-of-concept web application called Aphantasia, which utilizes such an approach, including 
a custom \gls{GLA} implementation and a rendering engine based on PixiJS.
During the development, we have solved several technical challenges regarding user management,
server-client communication, hosting, and, of course, graph rendering.

The finished application provides:
\begin{itemize}
    \item \textbf{Graph view} able to render several hundred nodes at once
    \item \textbf{Dynamic loading} ensuring the capability to explore large graphs in order of tens of thousands of nodes (and potentially much more)
    \item \textbf{User management} including registration and login
    \item \textbf{User interface}, including not just graph view but also pages with user settings, post creation form, chat
 and rudimentary notifications
   \item \textbf{Live preview} of the graph view
\end{itemize}

We are satisfied with the result and believe the application is a good starting point for further development.

We also believe that we have proven graph view as a viable alternative to interact with social media content,
provided that the content structure is allowed to have a form of \gls{DAG}.
Graph view might not be suitable for everyone, but it certainly attracts a niche of users who appreciate the exploration experience it provides.