%%% Please fill in basic information on your thesis, which will be automatically
%%% inserted at the right places. You need to replace \xxx{...} by real data.

% Type of your thesis:
%	"bc" for Bachelor's
%	"mgr" for Master's
%	"phd" for PhD
%	"rig" for rigorosum
\def\ThesisType{bc}

% Language of your study programme:
%	"cs" for Czech
%	"en" for English
\def\StudyLanguage{cs}

% Thesis title in English (exactly as in the official assignment)
% (Note: \xxx is a "ToDo label" which makes the unfilled visible. Remove it.)
\def\ThesisTitle{A Tool for Graph Visualisation of Social Networks}

% Author of the thesis (you)
\def\ThesisAuthor{Aleš Kakos}

% Year when the thesis is submitted
\def\YearSubmitted{2025}

% Name of the department or institute, where the work was officially assigned
% (according to the Organizational Structure of MFF UK in English,
% see https://www.mff.cuni.cz/en/faculty/organizational-structure,
% or a full name of a department outside MFF)
\def\Department{Department of Software Engineering}

% Is it a department (katedra), or an institute (ústav)?
\def\DeptType{Department}

% Thesis supervisor: name, surname and titles
\def\Supervisor{Doc. RNDr. Irena Holubová, Ph.D.}

% Supervisor's department (again according to Organizational structure of MFF)
\def\SupervisorsDepartment{Department of Software Engineering}

% Study programme (does not apply to rigorosum theses)
\def\StudyProgramme{Informatics}

% An optional dedication: you can thank whomever you wish (your supervisor,
% consultant, who provided you with tea and pizza, etc.)
\def\Dedication{%
I would like to thank my supervisor, Doc. RNDr. Irena Holubová, Ph.D. for her guidance and support
as well as the people making up the Software Engineering department at Charles University for giving me the opportunity to study the ever-so-interesting field of computer magic.

I would also like to thank my family for supporting me on this path of life.
Specical thanks belongs to my mother who didn't allow me to give up when things got tough and to my granfather to whom I promised the title of Bachelor bearing his surname.

Last but not least I want to thank my girlfriend and future wife for her patience and understanding during the hard times of my long overdue studies.
}

% Abstract (recommended length around 80-200 words; this is not a copy of your thesis assignment!)
\def\Abstract{
    This work is about the design and implementation of a website called Aphantasia - a social network for graph enthusiasts
    which allows users to create posts, link them to others, and view them as a part of a shared, ever-expanding interactive graph.
    This strategy presents an alternative to user-post interaction on social networks compared to the mainstream infinite scroll approach.
    We developed Aphantasia into a fully functional prototype and compared it to other graph-rendering software.
    Particular attention was paid to handling large numbers of posts without sacrificing the graph view's performance, readability, and user experience.
}

% 3 to 5 keywords (recommended) separated by \sep
% Keywords are useful for indexing and searching for the theses by topic.
\def\ThesisKeywords{%
graph data\sep visualization\sep Big Data
}

% If any of your metadata strings contains TeX macros, you need to provide
% a plain-text version for use in XMP metadata embedded in the output PDF file.
% If you are not sure, check the generated thesis.xmpdata file.
\def\ThesisAuthorXMP{\ThesisAuthor}
\def\ThesisTitleXMP{\ThesisTitle}
\def\ThesisKeywordsXMP{\ThesisKeywords}
\def\AbstractXMP{\Abstract}

% If your abstracts are long and do not fit in the infopage, you can make the
% fonts a bit smaller by this setting. (Also, you should try to compress your abstract more.)
\def\InfoPageFont{}
%\def\InfoPageFont{\small}  % uncomment to decrease font size

% If you are studing in a Czech programme, you also need to provide metadata in Czech:
% (in English programmes, this is not used anywhere)

\def\ThesisTitleCS{Nástroj pro grafickou vizualizaci sociálních sítí}
\def\DepartmentCS{Katedra softwarového inženýrství}
\def\DeptTypeCS{Katedra}
\def\SupervisorsDepartmentCS{Katedra softwarového inženýrství}
\def\StudyProgrammeCS{Informatika}

\def\ThesisKeywordsCS{%
grafová data\sep vizualizace\sep velká data
}

\def\AbstractCS{%
Tato práce se zabývá návrhem a implementací webové stránky nazvané Afantázie – sociální sítě pro grafové nadšence,
která umožňuje uživatelům vytvářet příspěvky, propojovat je s ostatními a zobrazit je jako rostoucí sdílený interaktivní graf.
Tento přístup představuje alternativu k interakci uživatelů s příspěvky na sociálních sítích v porovnání s přístupem nekonečného posouvání.
Afantázii jsme vyvinuli do stavu plně funkčního prototypu a porovnali ji s jiným softwarem pro vykreslování grafů.
Zvláštní pozornost jsme věnovali zobrazení velkého množství příspěvků, aniž bychom omezili výkon, čitelnost a uživatelskou přívětivost grafového zobrazení.
}
