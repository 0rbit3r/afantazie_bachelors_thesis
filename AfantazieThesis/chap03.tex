\chapter{Design Analysis}
\label{chap:design}

Before implementing Aphantasia, we will introduce its requirements and set expectations.
In short, Aphantasia will be a social network based on an Obsidian-like graph view.

\section{Intended Use}
The use case of Aphantasia is very similar to other social networks.
Users will be incentivized to create text-based posts called \glspl{thought}, link them to other thoughts, and try to collect replies from other users.

Apart from the graph aspect, Aphantasia will be novel in the structure of its content.
Whereas most other similar social networks allow only one link per post, Aphantasia will allow up to 5.
In graph theory, the structural difference between the two approaches is that while other networks contain forests,
Aphantasia will be made of \gls{DAG} components.

We expect this element to make the posts more interconnected.
On that basis, we want to create an online experience that encourages two aspects:
\begin{itemize}
  \item \textbf{Exploration} of the thoughts of others
  \item and creation of new \textbf{associations} between them.
\end{itemize}

The ability to reply to more than one thought at once combined with graph view means users will be able to 
interlink two separate connected components together and thus create bridges of associations.

\section{Functional Requirements}
The functional requirements of Aphantasia are as follows:

\subsection{Graph View}
The focus of Aphantasia is the graph view.
We want to create a comparable user experience to Obsidian's graph view, and thus, we will need to implement the following features:
\begin{itemize}
  \item \textbf{Node Rendering} - Each thought should be rendered as a node
  \item \textbf{Edge Rendering} - Links between thoughts should be rendered as edges
  \item \textbf{Animated Graph Layout} - The graph should stabilize using an FDL algorithm and animate the stabilization process
  \item \textbf{Interactivity} - Nodes should be draggable to influence the layout algorithm
  \item \textbf{Movement and Zoom} - A viewport with the ability to move around and zoom in and out through the graph
  \item \textbf{Content Preview} - Users should be able to click on a thought to see its content
  \item \textbf{Clickable links and backlinks} - The content preview needs to provide clickable links and backlinks (i.e., replies)
  \item \textbf{Floating Titles} - When zoomed in past a certain threshold, titles of thoughts should be displayed under their nodes in the graph view
\end{itemize}

To then elevate the experience beyond Obsidian and towards its own online shared experience, we will need to implement the following features:
\begin{itemize}
  \item \textbf{User-specific Coloring} - Thoughts in graph view should be colored according to the user who created them
  \item \textbf{Replies-dependent Node Size} - Thoughts that have more replies should appear larger to indicate their importance
  \footnote{Obsidian does have this feature as well, but it is not as pronounced as we would like}
  \item \textbf{Graph Exploration} - The graph should be explorable by traversing the thoughts through their edges
  \item \textbf{Dynamic loading} - Only a subset of thoughts should be loaded at once based on the user's position in the graph
  \item \textbf{Big Graph Support} - The graph view should be able to handle arbitrarily big graphs
\end{itemize}

\subsection{User Management}
Aphantasia needs to provide a simple user management system.
Concretely:
\begin{itemize}
  \item \textbf{Registration}
  \item \textbf{Login}
  \item \textbf{Logout}
  \item \textbf{Account Personalization}
\end{itemize}
The personalization aspect should include the ability to choose color. Users' thoughts will be displayed in this color.

\subsection{Thought Creation}
Registered users should be able to create new thoughts and link them to other existing thoughts.
Each thought should have a title, a body, and up to 5 links to other thoughts.

% The number of links is an important feature and a departure from mainstream social media, which usually allows only one link per post.
% \xxx{As noted in... ?}
% Most websites have the structure of a forest - each root node represents a post, and its children are comments.
% \xxx{Tweets and retweets}

\subsection{Routing and Pages}
Aphantasia should have at least the following pages:
\begin{itemize}
  \item \textbf{Graph View}
  \item \textbf{User Settings}
  \item \textbf{Post Creation Form}
  \item \textbf{Login and registration pages}
\end{itemize}

\section{Non-functional Requirements}
Our non-functional requirements will be focused mainly on the graph view and the user experience it provides.

\subsection{Graph View UX}
The graph view of Aphantasia should offer the following qualities:
\begin{itemize}
  \item \textbf{Fluidity} - The graph should be animated without stuttering or \gls{fps} drops
  \item \textbf{Responsive Design} - The graph view should be usable on both desktop and mobile devices
  \item \textbf{Performance} - The graph should be able to handle at least a hundred thoughts on screen at once
  \item \textbf{Intuitive Exploration} - The graph exploration should be intuitive and easy to understand
\end{itemize}

\subsection{Extendability}
We would like to design Aphantasia in a way that allows for easy extension and modification.
The app should be able to easily accommodate additional features such as:
\begin{itemize}
  \item \textbf{Notifications} - In the future, a page with replies to user's thoughts is planned
  \item \textbf{Search and Filters} - In time, users should be able to search for thoughts and filter them based on various criteria
  \item \textbf{Graph Parameters Modification} - The graph view parametrization (such as the strength of the forces in the FDL algorithm)
 does not need to be available to the users right away, but the application should allow for it in the future
\end{itemize}