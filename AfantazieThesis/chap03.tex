\chapter{Software design}

Before implementing Aphantasia we will introduce the application, set up expectations, decide on technologies used and plan the execution.

\section{Requirements and use-cases}

In short, Aphantasia can be described as an attempt to create web-based multi-user Obsidian.
It will allow users to collectively create posts, interlink them and view them in a graph view.

We are expecting to implement at least these following features:
\begin{itemize}
  \item User registration and login
  \item Post creation and linking
  \item Graph view of the posts
  \item Zooming and panning of the graph
  \item Large graph handling
  \item Graph filtering and search functionality
\end{itemize}


\section{Technologies and architecture}

Aphantasia will be a publicly accessible web application and as such we will need to set up the following:
\begin{itemize}
  \item Frontend - a webpage serving an interactive graph and UI
  \item Backend - for bussiness logic and data processing
  \item Database - for persistence of data
  \item Hosting - to make the website accessible
\end{itemize}

Let's start from the bottom with hosting.
We will use a Virtual Private Server (VPS) from a hosting company Váš hosting.
This will provide us with a linux server fully prepared for hosting including
SSL certificates, domains, databases and a lot of tools to manage the server.

For database will use PostgreSQL as it is one of the most popular database engines and is known for its reliability and performance.

For the backend we will use ASP.NET Core with C\#.
While the main reason for this decision is familiarity with the technology the .NET platform is regardles a good choice as:
\begin{itemize}
    \item It is linux compatible
    \item The provided Entity Framework library providesORM, complete Code-first solutions and makes it easy to replace database technology later on if needed.
    \item It is just good (cit. needed... statistics of usage by firms?)
\end{itemize}

Frontend is where the most interesting decisions will be made. It makes a lot of sense to use a modern frontend javascipt library.
We have experience with two - Angular and React.

We will use React.js with Typescript. React is a popular frontend library and Typescript is a superset of Javascript that provides static typing.
The reason for this choice is the fact that we will have to implement a rendering engine.
React has unopinionated approach to architecture which should make it easier to integrate with our rendering solution.

\xxx{todo - first more info about the different borad-strokes options we
have and why we decided for our own simultaion and rendering solution - simulation in BE vs FE etc.}

For rendering the graph we have two options - use an existing library or implement our own.
We looked into one such library that might be usable in our case - Cytoscape.js.
\xxx{But...}

We will implement our own simulation and rendering engine using a rendering library for javascript. The reasoning behind this is two-fold:
\begin{itemize}
    \item We will have more control over the rendering process and appearance of the graph
    \item Rendering graphs and simultaing FDL is not a too difficult of a task and we should be able to implement a simple and extendable version in a reasonable time frame.
\end{itemize}

We considered four possibilities:
\begin{itemize}
    \item \textbf{Html and css} - the most barebones solution would be to use a set of divs to represent nodes and svg lines to represent edges
    \item \textbf{HTML5 canvas} - a more sophisticated solution to draw both nodes and edges in HTML5 Canvas api
    \item \textbf{PIXI.js} - library for 2D rendering in the browser
    \item \textbf{Three.js} - library for 3D rendering in the browser
    \footnote{Three.js is capable of 2D rendering. However it is evident that 2D rendering is achieved by rendering objects 3D space and orthographic camera.} \xxx{cit?}
\end{itemize}

We chose to use \textbf{PIXI.js} as it seemed to provide good compromise between ease of use and performance for 2D rendering in the browser.

\section{Utilized algorithms}

We will use a force-directed layout algorithm to render the graph. It's easy to implement and provides visually pleasing results.
There as 

\section{Data}

\section{Plan of execution}
\begin{enumerate}
    \item set up the hosting and database
    \item build and host a simple web application in react
    \item create user management 
    \item implement a proof of concept of the graph engine for small graphs
    \item extend the application to handle large graphs
\end{enumerate}