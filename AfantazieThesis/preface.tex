\chapter*{Introduction}
\addcontentsline{toc}{chapter}{Introduction}

\section*{Motivation}

In recent years, it has been hard to escape the infinite feed.
It is a design pattern where the user is presented with a list of posts that can be scrolled through seemingly indefinitely.
Infinite feed is the primary way of presenting posts used by, for example
TikTok, Instagram, Facebook, Twitter, Reddit, and, to a lesser extent, YouTube.
This approach is easy to implement, intuitive, and requires minimal user interaction
- A swipe or scroll is all the input is needed.

While the infinite feed is not necessarily bad or insufficient, it does have some drawbacks. Namely:
\begin{itemize}
  \item \textbf{Echo chambers}
 - When the recommendation system behind the infinite feed also includes or even centers around user preferences as a factor,
 the ideological or political similarity of the content consumed can narrow the user's worldview.
 This effect is often called an "echo chamber" \cite{echo_chamber_wiki}.
  \item \textbf{Addictive design}
 - The infinite feed might encourage mindless, addictive consumption of content with each swipe akin to the pulling of a slot machine lever.
 The term "doomscrolling" \cite{doomscrolling_wiki} has been coined to describe this behavior.
  \item \textbf{Lack of autonomy}
 - The user does not decide on the next post in the feed. Instead, it is decided by the recommendation algorithm
 (colloquially referred to as "The algorithm" \cite{the_algorithm_wikitionary}),
 which can be opaque and hard for content creators and consumers to understand.
\end{itemize}

In this work, we will explore a graph-based approach as an alternative to the infinite feed pattern.

\section*{Aphantasia}
Aphantasia (sometimes also referred to as Afantázie) is the implementation part and the final product of this thesis,
currently available at:
\begin{itemize}
  \item \textbf{\url{https://aphantasia.io}} internationally
  \item \textbf{\url{https://afantazie.cz}} for Czech user base
\end{itemize}

It is a social network concept based on graph visualization of its contents.
It lets users create posts called thoughts, interlink them, and explore them as animated, interactive, and colorful graphs.

In the following text, we will go through the design and implementation of Aphantasia and
explore whether it can mitigate or at least alleviate the drawbacks of the infinite feed pattern.

\begin{itemize}
  \item We will start by introducing the concept of Graph Layout algorithms in Chapter \ref{chap:graph_layout_algorithms}.
  \item Next, we will compare software products that provide graph visualization in Chapter \ref{chap:related_software}.
  \item In Chapter \ref{chap:design}, we will look at the analysis, requirements, and use cases of Aphantasia.
  \item Then we will implement the software accordingly in Chapter \ref{chap:implementation}.
  \item Chapter \ref{chap:testing} is where we test Aphantasia.
 We will compare it to the software from Chapter \ref{chap:related_software},
 try its quantitative limits, and evaluate the user experience when compared to the infinite feed.
  \item And finally, in Chapter \ref{chap:documentation}, we will provide the User, Developer, and Administrator documentation.
\end{itemize}