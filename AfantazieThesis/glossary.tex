\newglossaryentry{graph}{
    name={graph},
    description={- a set of \textbf{nodes} connected by \textbf{edges}}
}
% \newglossaryentry{thought}{
%     name={thought},
%     description={ - a post on Aphantasia, has title, author, description, links and backlinks(also reffered to as replies), can be viewed as a node in a graph}
% }
\newglossaryentry{node}{
    name={node},
    description={- a single entity in a \textbf{graph}}
}
\newglossaryentry{edge}{
    name={edge},
    description={- a connection between two \textbf{nodes} in a \textbf{graph}}
}
\newglossaryentry{directed_edge}{
    name={directed edge},
    description={- an \textbf{edge} with a direction from one \textbf{node} to another}
}
\newglossaryentry{cycle}{
    name={cycle},
    description={- a set of \textbf{nodes} connected by \textbf{edges} in a way that the first and last node are the same}
}
\newglossaryentry{big_graph}{
    name={Big / Small graph},
    description={- for the purposes of this work big graphs are graphs containing at least thousand of \textbf{nodes}}
}
\newglossaryentry{thought}{
    name={thought},
    description={- a post on Aphantasia - it has title, author, description, links and backlinks(also reffered to as replies)}
}
% \newglossaryentry{DAG}{
%     name={DAG},
%     description={- directed acyclic graph}
% }
\newglossaryentry{GLA}{
    name={GLA},
    description={- graph Layout Algorithm - used for computing positions of \textbf{nodes} so that they make a nice or useful diagram}
}
\newglossaryentry{FDL}{
    name={FDL},
    description={- force-directed layout - a type of \textbf{GLA} that simulates physical forces between \textbf{nodes} to determine their positions}
}
\newglossaryentry{BFS}{
    name={BFS},
    description={- breath first search - algorithm for traversing a \textbf{graph} while prioritizing earlier visited \textbf{nodes}}
}
\newglossaryentry{path}{
    name={path},
    description={- in context of \textbf{graphs}, path is a sequence of \textbf{nodes} connected by \textbf{edges} without any \textbf{cycles}}
}
\newglossaryentry{cithep_dataset}{
    name={CitHep},
    description={- dataset of citations between papers in the field of high-energy physics. Contains 34546 \textbf{nodes}, 421578 \textbf{edges} and temporal data}
}
\newglossaryentry{fps}{
    name={FPS},
    description={- frames per second - a measure of fluidity of animation and speed of a simulation}
}
\newglossaryentry{sha256}{
    name={SHA256},
    description={- cryptographic hash function commonly used to securely hash passwords and tokens}
}
\newglossaryentry{jwt}{
    name={JWT},
    description={- JSON Web Tokens, a compact and self-contained way to securely transmit information between parties as a JSON object}
}
\newglossaryentry{local_storage}{
    name={local storage},
    description={- web storage feature that allows JavaScript to store and retrieve data in the browser betwen sessions}
}
\newglossaryentry{vite}{
    name={Vite},
    description={- frontend build tool that provides configuration and optimized production builds}
}
\newglossaryentry{zustand}{
    name={Zustand},
    description={- state management library for React, allows better integration with external (ie. non-react) libraries}
}
\newglossaryentry{localization}{
    name={localization},
    description={- adaption of a product to a specific locale or market (language, currency, time format etc.)}
}
\newglossaryentry{production}{
    name={production},
    description={(environment) - the instance of an application that is accessible to its userbase, as opposed to development or testing environments}
}
\newglossaryentry{ORM}{
    name={ORM},
    description={- Object-Relational Mapping - a programming technique for converting data between backend and database}
}
\newglossaryentry{code_first}{
    name={code-first},
    description={- a development approach where the database schema is generated from the code, as opposed to the database-first approach}
}
\newglossaryentry{auth_x_y}{
    name={authX/Y},
    description={- authentication and authorization, set of systems verifying user identity and granting access to resources}
}
\newglossaryentry{dependency_injection}{
    name={dependency injection},
    description={- a programming technique in which class instances resolve their dependencies using a DI container set up during the startup}
}
\newglossaryentry{business_logic}{
    name={business logic},
    description={- the code that implements the business rules of a program (ie. what the application is designed for)}
}
\newglossaryentry{websockets}{
    name={websockets},
    description={- a two-way communication protocol allowing server sending messages to the client without client requesting them}
}
\newglossaryentry{memory_leak}{
    name={memory leak},
    description={- a situation where a program fails to release memory it no longer needs, leading to wasting memory and potentialy lower performance}
}

\printglossary
% \xxx{TODO - add a space before description if possible (Or is it okay to leave the dashes?)}
\addcontentsline{toc}{chapter}{Glossary}