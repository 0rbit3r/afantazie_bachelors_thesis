\chapter{The completed application}
In the grand scheme of a "real" social network Afantázie is quite lacking as it still missing a lot of obvious features.
Namely at least:
\begin{itemize}
    \item Email verification 
    \item Password reset
    \item Streamlined UI and UX
\end{itemize}

Despite that we believe our efforts did pay off.
The application does what we expected and more.

\section{Features}
Let us take a deeper look at the features the final application provides.

\subsection{Time slider}

\subsection{Live preview}

\subsection{Graph walk}
The neighborhood API endpoint mentioned above is used for a feature we call graph walk.
It's a simple idea - user can click on a node and the graph will load the neighborhood of the node.

This allows users to traverse the graph across associations and explore the data in a interactive way.
The nodes that blink also indicate that there is something to explore when some of their neighbors are not currently on screen.

\subsection{Realtime chat}
The chat room implemented in the very first stage of development remained as an extra feature.
While at this point chat is a completely separate feature from the graph, in the future it might be integrated into the graph more.
For example by providing the ability to link thoughts in-chat.


\section{Usage}
While not yet logged in users can acces the Landing page, the About page and the Graph page (in a slightly limited form) and of course the Login and registration pages.

After logging in users gain further access to
\begin{itemize}
    \item User settings
    \item Notifications
    \item Chat
\end{itemize}

\subsection{Graph view}
Graphs view provides several ways of exploration:
\begin{itemize}
    \item neighborhood of a highlighted thought
    \item thoughts in similar timeframe
    \item thoughts filtered by user-defined parameters such as author \xxx{or..?}
\end{itemize}

While we did discuss some of our graph view limits and quantitative parameters it is an interactive experience and as such is hard to describe without acknowleging it.
Allow us then to have a few subjective claims.
First, the pros:
\begin{itemize}
    \item The result is visualy pleasing, tactile and interactive in a very similar way to obsidians graph view, if not more
    \item Performance-wise the graph runs fairly smoothly on all tested desktop and mobile devices (with the default 300 thoughts on screen limit)
\end{itemize}
And the cons:
\begin{itemize}
    \item 
    \item 
\end{itemize}



\subsubsection*{Deviation from the original plan}
We did deviate from our initial plan of rendering big graphs.
In the beginnig we expected to implement a zoom-based dynamic loading and rendering.
But we ended up with time-based and proximity-based approach (ie. Time slider and neighborhood).

The reason behind this decision was mainly the ease of implementation.
While the zoom-based technique still remains a valid possibility it is more complex and could prove to be a roadblock.
The time slider also unexpectedly but naturaly emerged as a consequence of thoughts-on-screen-limit feature making a second way of handling big thoughts redundant time sink.
