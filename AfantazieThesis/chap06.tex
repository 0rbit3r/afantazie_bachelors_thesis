\chapter{Documentation}

\section{User documentation}
All of the pages are accessible from the homepage and/or the collapsable navbar in the top right.

\subsection{Registering and logging in}
While not logged in only graph view, about page and homepage are accessible.

To register click on the register button on the homescreen. The registration requires selecting a unique username, email and password.
The password has security requirements which are displayed immediately after opening the form and in case of not meeting the criteria on submit.

To log in click on the login button on the homescreen. The login requires a username or email and password.

At this point the email is not used anywhere and is only included in preparation for email verification feature.

\subsection{Opening a thought}
There are multiple ways to open a thought:
\begin{itemize}
  \item \textbf{New thoughts log on homepage} - On homepage there is a feed of the last three thoughts created on the website.
  Clicking on one of them will open the thought in graph view.
  
  Clicking on the "All Thoughts" button under the feed leads to all thoughts list. (This feature lacks paging and is to be improved in the future \xxx{todo?})
  \item \textbf{Notifications} - The Notifications button and the bell icon in the navbar lead to the list of replies.
  Replies are thoughts of other users that linked to any of the logged-in user's thoughts.
  \item \textbf{Graph view} - The main way to access thoughts is through the graph view. We will take a closer look at it in the next section.
  \item \textbf{Direct link} - Every thought has a unique ID which can be shared and accessed directly using the URL in format '/graph/\{thoughtId\}'.

  Example of the full URI leading to the thought with ID 1: https://afantazie.cz/graph/1
\end{itemize}

\subsection{Graph view}
To access graph view either click the "Graph" button on the main page or or click on the second icon in the navbar (three conected nodes). \xxx{figure?}

In graph view you can use the following controls:
\begin{itemize}
  \item \textbf{Mouse wheel or bottom right buttons} - Zooms in and out

  Once zoomed past a threshold titles of the thoughts appear.
  \item \textbf{Draging the background} - Pans the viewport
  \item \textbf{Dragging a node} - Moves the grabbed thought around
  
  This is useful for customizing the layout, "untying" thoughts that are too close to each other or to speed up the process of the layout algorithm. 
  \item \textbf{Clicking a thought} - Highlights a thought and switches to highlighted mode. (more info below)
\end{itemize}

\subsubsection*{Highlighted mode}

In default mode the whole display is used to view the graph and the user can interact with it as described above.
When a thought is accessed (either by clicking on it or opening it directly through the URL) \xxx{todo - describe this feature} the graph view switches to highlighted mode.

In highlighted mode half of the screen gets dedicated to the highlighted thought preview and the other half to the graph view. \xxx{figures}

The graph view in highlighted mode shares almost all behavior with the default mode with a few exceptions:
\begin{itemize}
  \item \textbf{Visual node highlight} - The currently opened thought is also visually highlighted by a white circle \xxx{(todo - figure)}
  
  \item \textbf{Visual edges highlight} - All edges connected to the opened thought are exaggerated in thickness and color while all other egdes are dimmed.
  \item \textbf{Neighborhood thoughts} - On highlighting a thought the application loads its neighborhood. \xxx{figure and see more below?}
\end{itemize}
This behavior helps users to distinguish highlighted thought from the others.
Every time a thought is selected the viewport also smoothly centers on it.

The thought preview shows title, author, time of creation, content with clickable links and replies section. \xxx{figure?}

Both the links in content and titles in replies section are color-coded based on authors selected color and on click will highlight the respective thought.

At the bottom of the preview there are Reply button and a close button (up icon on mobile and X on desktop).

On mobile devices a down arrow is also included which stretches the preview half of the screen to the full height
hiding the graph view and giving more screen real state to the textful information.

\subsubsection*{Neighborhood thoughts and graph walk}

When a thought is highlighted the application loads its neighborhood.
The neighborhood is defined as thoughts accessible through BFS \xxx{todo glossary} up to a given depth.
The currently used depth of the search is fixed at 3.

Some of the thoughts rendered on screen can be filled with black color. \xxx{todo figure}
This indicates that some neighbors of the node are not visible on screen and thus the node is explorable.
  
\subsection{Creating a new thought}

There are two ways to access the create thought page:
\begin{itemize}
  \item \textbf{From the graph view} - Click on the "New thought" button on the bottom of the graph view.
  \item \textbf{From the thought preview} - Click on the "Reply" button at the bottom of the thought preview.
\end{itemize}

Both of these ways lead to the Create thought page.
The difference between them is that when accessed through the Reply button the respective thought link is automatically added to the content input.

The Create thought page consists of two text fields: Title and Content.

Both the title and content are required. Title has a minimum length of 1 character. The content's minimum length is 5 characters.
Both of these requirements are enforced by validation rules and the user is notified by notification messages if the submit fails.

\subsubsection*{Linking other thoughts}

\subsection{Settings}

\section{Developer documentation}

\subsection{Frontend}

\subsection{Backend}

\subsection{Database}

\subsection{Deployment}